%\documentclass{article}
\documentclass{beamer}

\usepackage{amsmath}
\usepackage[mathletters]{ucs}
\usepackage[utf8x]{inputenc}
\usepackage{graphicx}
%\usepackage[breaklinks=true,unicode=true]{hyperref}
%\setlength{\parindent}{0pt}
%\setlength{\parskip}{6pt plus 2pt minus 1pt}
%\setcounter{secnumdepth}{0}

\usetheme{Warsaw}

\title{LR parsing is the derivative of context free grammars}
\author{Josef Svenningsson}

\begin{document}

\begin{frame}
  \titlepage
\end{frame}

\begin{frame}
\frametitle{LR parsing is the derivative of context free grammars}

The title says it all.
\vfill
\pause
But you might not be convinced by me just saying so..
\end{frame}

\begin{frame}
\frametitle{Jogging your brains}

\begin{figure}[h]
\centering
\includegraphics{Brain_Runner_300.JPG}
\end{figure}
\end{frame}

\begin{frame}
\frametitle{Context Free Grammars}

A context free grammar
\vfill
$\begin{array}{lcl} E & \rightarrow & E * B\\
E & \rightarrow & E + B\\
E & \rightarrow & B\\
B & \rightarrow & 0\\
B & \rightarrow & 1\\ 
\end{array}$

\end{frame}

\begin{frame}
\frametitle{Formal Definition}

A Context Free Grammar is a four-tuple:

\begin{itemize}
\item
  A set of terminals $T$
\item
  A set of nonterminals $NT$
\item
  A start symbol $S \in NT$
\item
  A set of productions. Each production consist of:
  \begin{itemize}
  \item
    A nonterminal; the head
  \item
    The symbol $\rightarrow$
  \item
    A body consisting of a sequence of terminals and nonterminals
  \end{itemize}
\end{itemize}
\end{frame}

\begin{frame}[fragile]
\frametitle{Semantics of a grammar}

The semantics of a grammar is defined in terms of derivation
trees.

\begin{verbatim}
         E               E
         |               |
        /|\             /|\
       E * E           E + E
      /     \         /     \
     /|\     0       0     /|\
    0 + 1                 1 * 0
\end{verbatim}

If a string has more than one derivation the grammar is ambiguous.

\end{frame}

\begin{frame}
\frametitle{LR parsing}

\begin{itemize}
\item
  The most common algorithm for parsing formal languages.
\item
  Invented by Knuth
\item
  Lots of tools for generating LR parsers: YACC, Bison, Happy,
  \ldots{}
\end{itemize}
\end{frame}

\begin{frame}[fragile]
\frametitle{LR parser machine}

\begin{figure}[h]
\centering
\includegraphics[width=8 cm]{LR_Parser.png}
\end{figure}
\end{frame}

\begin{frame}
\frametitle{Producing the LR automaton}

Going from a grammar to an LR automaton goes via item sets.

\vfill

$\begin{array}{lcl} 
\multicolumn{3}{l}{\text{Item Set 0}}\\
E & \rightarrow & \bullet E * B\\
E & \rightarrow & \bullet E + B\\
E & \rightarrow & \bullet B\\
E & \rightarrow & \bullet 0\\
E & \rightarrow & \bullet 1\\
\\
\multicolumn{3}{l}{\text{Item Set 1}}\\
B & \rightarrow & 0 \bullet\\
\\
\multicolumn{3}{l}{\text{Item Set 2}}\\
B & \rightarrow & 1 \bullet\\
\end{array}$
\end{frame}

\begin{frame}
$\begin{array}{lcl}
\multicolumn{3}{l}{\text{Item Set 3}}\\ E & \rightarrow & E \bullet * B\\ E & \rightarrow & E \bullet + B\\ \end{array}$

\vfill

Advancing the dot one step is like recognizing one character in the
input, alternatively a whole production.
\end{frame}

\begin{frame}
\frametitle{Item set}

Each item set corresponds to a state in the LR automaton (for LR(0)
parsers).
\end{frame}

\begin{frame}
\frametitle{Item sets are the derivative of the context free grammar}

My claim:

Item sets are the derivative of the context free grammar
\end{frame}

\begin{frame}
\frametitle{A new representation of Context Free Grammars}

It's not exactly clear from the presentation of item sets that they
correspond to some kind of derivative

To demonstrate this we need a new representation of context free
grammars
\end{frame}

\begin{frame}
\frametitle{A new syntax for syntax}

$\begin{array}{lcl} G,H \in \text{Grammar} & ::= & x \: | \: 0 \: | \: 1 \: | \: G \cdot H \: | \: G \: + \: H \: | \: T \: A\\ T \in \text{Grammar Tuples} & ::= & r \: | \: \mu r . \{ A \mapsto G \}_{A \in N} \end{array}$
\end{frame}

\begin{frame}
\frametitle{An example grammar}

$\begin{array}{lll} \mu r . \{ & E \mapsto (r E \: \text{'+'} \: r B) + (r E \: \text{'*'} \: r B) + r B ; \\ & B \mapsto 0 + 1 & \}_{\{E, B\}} \end{array}$
\vfill
$\begin{array}{lcl} E & \rightarrow & E * B\\ E & \rightarrow & E + B\\ E & \rightarrow & B\\ B & \rightarrow & 0\\ B & \rightarrow & 1\\ \end{array}$
\end{frame}

\begin{frame}
\frametitle{Semantics}

\begin{itemize}
\item The current semantics I have is unfortunately not based on
derivation trees.

\item It only captures whether a grammar recognizes a language or not.
There is no notion of ambiguity.

\item $0, 1, + \: \text{and} \: *$ form a semi-group
\end{itemize}
\end{frame}

\begin{frame}
\frametitle{The Derivative}

$\begin{array}{lcl} \partial_x x & = & 1 \\ \partial_x (y\not= x) & = & 0 \\ \partial_x 0 & = & 0 \\ \partial_x 1 & = & 0 \\ \partial_x (G \cdot H) & = & \partial_x G \cdot H + \delta G \cdot \partial_x H \\ \partial_x (G + H) & = & \partial_x G + \partial_x H \\ \partial_x (r A) & = & 0 \\ \partial_x ((\mu r . \{ A \mapsto G_A\}_{A \in N}) B) & = & \mu r . \{ S \mapsto \partial_x G_B \}_{\{S\}\cup N} S \end{array}$
\end{frame}

\begin{frame}

\frametitle{Auxiliary function, empty string recognition}

$\begin{array}{lcl}
\delta x & = & 0\\
\delta 0 & = & 1\\
\delta 1 & = & 0\\
\delta (G \cdot H) & = & \delta G \cdot \delta H\\
\delta (G + H) & = & \delta G + \delta H\\
%\delta (T A) & = & (\delta T) A\\
\delta (r A) & = & r A\\
\delta (\mu r . \{A \mapsto H_A\}_N A) & = & \mu r . \{ A \mapsto \delta H_A\} A
\end{array}$
\end{frame}

\begin{frame}

\frametitle{Derivative wrt Non-terminals}

$\begin{array}{lcl}
\partial_A x & = & 0\\
\partial_A 0 & = & 0\\
\partial_A 1 & = & 0\\
\partial_A (G \cdot H) & = & \partial_A G \cdot H + \delta G \cdot \partial_A H\\
\partial_A (G + H) & = & \partial_A G + \partial_A H\\
\partial_A (r A) & = & 1\\
\partial_A (r B\not= A) & = & 0\\
\partial_A (\mu r . \{B \mapsto G_B\}_N B) & = & \mu r . \{ B \mapsto \partial_A G_B \}_N B\\
\end{array}$
\end{frame}

%\begin{frame}
%\frametitle{Translation}
%\end{frame}

\begin{frame}
\frametitle{The Proof}

\begin{figure}[h]
\centering
\includegraphics[width=100mm]{Diagram.png}
\end{figure}
\end{frame}

\begin{frame}
\frametitle{Derivatives}

The intuition behind derivatives in computer science: removing one
item

\begin{itemize}
\item
  Data types: making one hole
\item
  Regular expressions: recognizing one character
\item
  LR parsing: recognizing one character
\end{itemize}
\end{frame}

\begin{frame}
\frametitle{Questions}

Questions?
\end{frame}

\begin{frame}
\frametitle{The sequence rule}

$\partial_x (G \cdot H) = \partial_x G \cdot H + \delta G \cdot \partial_x H$
\vfill
$\partial_x (G \cdot H) = \partial_x G \cdot H + G \cdot \partial_x H$
\end{frame}

\end{document}

